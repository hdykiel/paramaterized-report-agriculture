\documentclass[]{article}
\usepackage{lmodern}
\usepackage{amssymb,amsmath}
\usepackage{ifxetex,ifluatex}
\usepackage{fixltx2e} % provides \textsubscript
\ifnum 0\ifxetex 1\fi\ifluatex 1\fi=0 % if pdftex
  \usepackage[T1]{fontenc}
  \usepackage[utf8]{inputenc}
\else % if luatex or xelatex
  \ifxetex
    \usepackage{mathspec}
  \else
    \usepackage{fontspec}
  \fi
  \defaultfontfeatures{Ligatures=TeX,Scale=MatchLowercase}
\fi
% use upquote if available, for straight quotes in verbatim environments
\IfFileExists{upquote.sty}{\usepackage{upquote}}{}
% use microtype if available
\IfFileExists{microtype.sty}{%
\usepackage{microtype}
\UseMicrotypeSet[protrusion]{basicmath} % disable protrusion for tt fonts
}{}
\usepackage[margin=1in]{geometry}
\usepackage{hyperref}
\hypersetup{unicode=true,
            pdftitle={Agricultural Trends},
            pdfauthor={Hadrien},
            pdfborder={0 0 0},
            breaklinks=true}
\urlstyle{same}  % don't use monospace font for urls
\usepackage{graphicx,grffile}
\makeatletter
\def\maxwidth{\ifdim\Gin@nat@width>\linewidth\linewidth\else\Gin@nat@width\fi}
\def\maxheight{\ifdim\Gin@nat@height>\textheight\textheight\else\Gin@nat@height\fi}
\makeatother
% Scale images if necessary, so that they will not overflow the page
% margins by default, and it is still possible to overwrite the defaults
% using explicit options in \includegraphics[width, height, ...]{}
\setkeys{Gin}{width=\maxwidth,height=\maxheight,keepaspectratio}
\IfFileExists{parskip.sty}{%
\usepackage{parskip}
}{% else
\setlength{\parindent}{0pt}
\setlength{\parskip}{6pt plus 2pt minus 1pt}
}
\setlength{\emergencystretch}{3em}  % prevent overfull lines
\providecommand{\tightlist}{%
  \setlength{\itemsep}{0pt}\setlength{\parskip}{0pt}}
\setcounter{secnumdepth}{0}
% Redefines (sub)paragraphs to behave more like sections
\ifx\paragraph\undefined\else
\let\oldparagraph\paragraph
\renewcommand{\paragraph}[1]{\oldparagraph{#1}\mbox{}}
\fi
\ifx\subparagraph\undefined\else
\let\oldsubparagraph\subparagraph
\renewcommand{\subparagraph}[1]{\oldsubparagraph{#1}\mbox{}}
\fi

%%% Use protect on footnotes to avoid problems with footnotes in titles
\let\rmarkdownfootnote\footnote%
\def\footnote{\protect\rmarkdownfootnote}

%%% Change title format to be more compact
\usepackage{titling}

% Create subtitle command for use in maketitle
\newcommand{\subtitle}[1]{
  \posttitle{
    \begin{center}\large#1\end{center}
    }
}

\setlength{\droptitle}{-2em}

  \title{Agricultural Trends}
    \pretitle{\vspace{\droptitle}\centering\huge}
  \posttitle{\par}
    \author{Hadrien}
    \preauthor{\centering\large\emph}
  \postauthor{\par}
      \predate{\centering\large\emph}
  \postdate{\par}
    \date{3/27/2019}


\begin{document}
\maketitle

\hypertarget{overview}{%
\section{Overview}\label{overview}}

This analysis aims to better understand agricultural production trends
in order to better guide decisions related to the manufacture of farming
equipment to ensure future food security in different geographies.

\textbf{Data:} Data for this analysis was obtained from the
\href{http://www.fao.org/faostat/en/\#data/QC}{Food and Agricultural
Organization of the United Nations} website on April 1st, 2019. Data was
downloaded as the bulk download ``All Data''.

\textbf{Using this report}

To view this report, go to the report URL on RStudio Connect. On the
left hand side of the report, select your desired input paramaters and
click run. A report can be generated for any country \& crop
combination. This report was designed for the following users:

\begin{itemize}
\tightlist
\item
  Marketing managers: explore production trends to plan which markets to
  invest in and create campaigns for.
\item
  Product managers: identify which product lines to invest in to help
  farmers overcome challenges such as droughts and desertification and
  meet the demand of an ever increasing population. K
\end{itemize}

\textbf{Contributing}

To contribute to this project, fork this report's repository on
\href{https://github.com/hdykiel/paramaterized-report-agriculture/commits/master}{github}
and create a new branch using the format ``your-name-feature-name'' and
submit a pull request.

\hypertarget{worldwide-production-of-wheat}{%
\subsubsection{Worldwide production of
Wheat}\label{worldwide-production-of-wheat}}

\includegraphics{agriculture-stats-paramaterized-inputs_files/figure-latex/plot1-1.pdf}

\includegraphics{agriculture-stats-paramaterized-inputs_files/figure-latex/plot2-1.pdf}

\hypertarget{production-trend-mexico}{%
\subsubsection{Production trend
(Mexico)}\label{production-trend-mexico}}

\href{https://en.wikipedia.org/wiki/Mexico}{Mexico} is a country in
Latin America \& Caribbean with an estimated population of
1.1121179\times 10\^{}\{8\}. Its government system is Sovereign country
and is categorized as having a ``4. Emerging region: MIKT'' economy.


\end{document}
